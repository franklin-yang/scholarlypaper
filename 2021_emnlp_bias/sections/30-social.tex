\section{Ambiguity: A social science perspective}
We adopt ideas from the social sciences which demonstrate ambiguity as a revelatory mechanism for bias. 
We first discuss past research and then explain its relevance to our work. 

Ambiguous questions, which lack a clear answer or have multiple possible answers, force individuals to rely on unconscious biases and self-serving traits~\cite{dunning1989ambiguity} due to the lack of structure, allowing for leeway in its interpretation. Prior work has shown when asked to rate themselves along various measures, individuals are more likely to overrate ambiguous abilities.
When answering ambiguous questions, people select the interpretation which makes them look best~\cite{dunning1989ambiguity,bradley1978self}, as shown through studies involving psychology students~\cite{dunning1989ambiguity}, football players~\cite{felson1981ambiguity}, and anxious subjects~\cite{eysenck1991bias}. 
In legal settings, ambiguous evidence can lead jurors to rely on implicit biases rather than evidence to make decisions~\cite{levinson2009different}.
Ambiguity serves as a modal to explore what factors people and systems use to make choices when allowed more freedom in the absence of certainty~\cite{felson1981ambiguity}.
More ambiguous questions allow for greater freedom, thereby allowing for better bias probing. 
Therefore, we develop in Section ~\ref{sec:bias} two types of ambiguous questions with varying degrees of freedom used in our experiments.